\documentclass[11pt]{article}
\usepackage[utf8]{inputenc}	% Para caracteres en español
\usepackage{amsmath,amsthm,amsfonts,amssymb,amscd}
\usepackage{multirow,booktabs}
\usepackage[table]{xcolor}
\usepackage{fullpage}
\usepackage{lastpage}
\usepackage{enumitem}
\usepackage{fancyhdr}
\usepackage{mathrsfs}
\usepackage{wrapfig}
\usepackage{setspace}
\usepackage{calc}
\usepackage{multicol}
\usepackage{cancel}
\usepackage[retainorgcmds]{IEEEtrantools}
\usepackage[margin=1cm]{geometry}
\usepackage{amsmath}
\newlength{\tabcont}
\setlength{\parindent}{0.0in}
\setlength{\parskip}{0.05in}
\usepackage{empheq}
\usepackage{framed}
\usepackage[most]{tcolorbox}
\usepackage{xcolor}
\usepackage{graphicx}
\usepackage{listings}
% -- Basic formatting
\usepackage[utf8]{inputenc}
\usepackage[english]{babel}
\usepackage{times}
\usepackage{caption}
\usepackage{subcaption}
\usepackage{placeins}
\setlength{\parindent}{0pt}
\usepackage{indentfirst}% -- Defining colors:
\usepackage[dvipsnames]{xcolor}
\definecolor{codegreen}{rgb}{0,0.6,0}
\definecolor{codegray}{rgb}{0.5,0.5,0.5}
\definecolor{codepurple}{rgb}{0.58,0,0.82}
\definecolor{backcolour}{rgb}{0.95,0.95,0.92}% Definig a custom style:
\lstdefinestyle{mystyle}{
    backgroundcolor=\color{backcolour},   
    commentstyle=\color{codepurple},
    keywordstyle=\color{NavyBlue},
    numberstyle=\tiny\color{codegray},
    stringstyle=\color{codepurple},
    basicstyle=\ttfamily\footnotesize\bfseries,
    breakatwhitespace=false,         
    breaklines=true,                 
    captionpos=t,                    
    keepspaces=true,                 
    numbers=left,                    
    numbersep=5pt,                  
    showspaces=false,                
    showstringspaces=false,
    showtabs=false,                  
    tabsize=2
}% -- Setting up the custom style:
\lstset{style=mystyle}
\lstset{
  style=mystyle,
  framexleftmargin=3.5mm,
  rulesepcolor=\color{black},
  linewidth=0.6\linewidth,
  xleftmargin=12pt,
  aboveskip=12pt,
  belowskip=12pt
}
\colorlet{shadecolor}{orange!15}
\parindent 0in
\parskip 1pt
\geometry{margin=1in, headsep=0.25in}
\theoremstyle{definition}
\newtheorem{defn}{Definition}
\newtheorem{reg}{Rule}
\newtheorem{exer}{Exercise}
\newtheorem{note}{Note}
\graphicspath{ {./images/} }
\begin{document}
\setcounter{section}{0}
\title{MIE223 Lecture Notes}

\thispagestyle{empty}

\begin{center}
{\LARGE \bf Guest Lecture}\\
{\large MIE223}\\
Winter 2025
\end{center}
\section{Data Science at Walmart}
\subsection{Introduction}
Senior data scientist for Walmart Ecommerce.

\subsection{Why Data Science}
\begin{itemize}
    \item passion in applied ML projects in undergrad
    \item Data Scientist at Citibank
    \item Worked with Sanner on clustering and data science projects
\end{itemize}

\subsection{What I love about DS at Walmart}
\begin{itemize}
    \item creating value for both customers and Walmart
    \item Witness real-time impact of your work
    \item Engaging problem statements and innovative environment
    \item perfect balance of individual contribution and collaboration
\end{itemize}

\section{Data Science in Retail}
\subsection{Forecasting}
\begin{itemize}
    \item How integral is demand forecasting in retail?
    \item Inventory planning
    \item Financial budgeting
    \item Workforce management
    \item Waste management
\end{itemize}

\subsection{Personalization}
\begin{itemize}
    \item ML powered recommedation engine
    \item Product listing to cater to needs of the customers
    \item Simple cross-sell items based on category relationships
\end{itemize}

\subsection{Product attribute extraction}
\begin{itemize}
    \item why does AI powered recommendation system work better
    \item Richer product descriptions
    \item Better understanding of customer needs
\end{itemize}

\section{Day in the life}
\subsection{Define the problem statement}
\begin{itemize}
    \item meet with business stakeholders to understand their requirements
    \item Create a rough timeline with a simple deliverable and follow up meeting
    \item Use case: demand forecasting for Walmart stores
\end{itemize}

\begin{itemize}
    \item Business requirement: workforce management
    \item Alignment: time series model that forecasts for orders per day at the store level
\end{itemize}

\subsection{Data Pipeline}
\begin{itemize}
    \item lot of time goes into data preparation
    \item Build SQL query worklflows to bring the data into the data lake
    \item SQL workflows can be very complex
    \item Requires joining data fro various tables to arrive at a simple tabular dataset
\end{itemize}

\subsection{Exploratory data analysis}
\begin{itemize}
    \item explore feasibility of the problem statement
    \item understand the data with simple analysis (missing data, cold-start problem...)
\end{itemize}

\subsection{Modelling and Experimentation}
\begin{itemize}
    \item Most exciting phase of the projects
    \item Try out innovative modelling ideas
    \item Run experiments to compare models
    \item Exploration vs producing an MVP
\end{itemize}

\subsection{Running pilot/test programs}
\begin{itemize}
    \item real-time impact of your work
    \item anxious waiting period where you want to achieve "better" results
    \item A successful pilot program leads to a full rollout
    \item A/B Testing
\end{itemize}

\subsection{Reporting}
\begin{itemize}
    \item monitor the model performance and provide summarized results to the stakeholders
    \item business emphasizes on explainability of the models
    \item Business emphasizes on:
    \begin{itemize}
        \item Explainability of the models
        \item Quick results
    \end{itemize}
\end{itemize}

\subsection{Model enhancements}
\begin{itemize}
    \item univariate versus multivariate time series modelling
    \item can use features like weather or store level intiatives to model the time series
\end{itemize}


\end{document}