\documentclass[11pt]{article}
\usepackage[utf8]{inputenc}	% Para caracteres en español
\usepackage{amsmath,amsthm,amsfonts,amssymb,amscd}
\usepackage{multirow,booktabs}
\usepackage[table]{xcolor}
\usepackage{fullpage}
\usepackage{lastpage}
\usepackage{enumitem}
\usepackage{fancyhdr}
\usepackage{mathrsfs}
\usepackage{wrapfig}
\usepackage{setspace}
\usepackage{calc}
\usepackage{multicol}
\usepackage{cancel}
\usepackage[retainorgcmds]{IEEEtrantools}
\usepackage[margin=1cm]{geometry}
\usepackage{amsmath}
\newlength{\tabcont}
\setlength{\parindent}{0.0in}
\setlength{\parskip}{0.05in}
\usepackage{empheq}
\usepackage{framed}
\usepackage[most]{tcolorbox}
\usepackage{xcolor}
\usepackage{graphicx}
\usepackage{listings}
% -- Basic formatting
\usepackage[utf8]{inputenc}
\usepackage[english]{babel}
\usepackage{times}
\usepackage{caption}
\usepackage{subcaption}
\usepackage{placeins}
\setlength{\parindent}{0pt}
\usepackage{indentfirst}% -- Defining colors:
\usepackage[dvipsnames]{xcolor}
\definecolor{codegreen}{rgb}{0,0.6,0}
\definecolor{codegray}{rgb}{0.5,0.5,0.5}
\definecolor{codepurple}{rgb}{0.58,0,0.82}
\definecolor{backcolour}{rgb}{0.95,0.95,0.92}% Definig a custom style:
\lstdefinestyle{mystyle}{
    backgroundcolor=\color{backcolour},   
    commentstyle=\color{codepurple},
    keywordstyle=\color{NavyBlue},
    numberstyle=\tiny\color{codegray},
    stringstyle=\color{codepurple},
    basicstyle=\ttfamily\footnotesize\bfseries,
    breakatwhitespace=false,         
    breaklines=true,                 
    captionpos=t,                    
    keepspaces=true,                 
    numbers=left,                    
    numbersep=5pt,                  
    showspaces=false,                
    showstringspaces=false,
    showtabs=false,                  
    tabsize=2
}% -- Setting up the custom style:
\lstset{style=mystyle}
\lstset{
  style=mystyle,
  framexleftmargin=3.5mm,
  rulesepcolor=\color{black},
  linewidth=0.6\linewidth,
  xleftmargin=12pt,
  aboveskip=12pt,
  belowskip=12pt
}
\colorlet{shadecolor}{orange!15}
\parindent 0in
\parskip 1pt
\geometry{margin=1in, headsep=0.25in}
\theoremstyle{definition}
\newtheorem{defn}{Definition}
\newtheorem{reg}{Rule}
\newtheorem{exer}{Exercise}
\newtheorem{note}{Note}
\graphicspath{ {./images/} }
\begin{document}
\setcounter{section}{0}
\title{MIE223 Lecture Notes}

\thispagestyle{empty}

\begin{center}
{\LARGE \bf Advanced Data Science and Fallacies}\\
{\large MIE223}\\
Winter 2025
\end{center}

\section{Potential Data Interpretation Fallacies}
\subsection{Anscombe Quartet: Visualize}
When we look at mean of x, sample variance of x, mean of y, sample variance of y,
correlation between x and y, and linear regression line, the accuracy differs.

If datasets have varying levels of accuracy, the mean, variance, and correlation will be the same. However, the linear regression line will differ.

\includegraphics[width=\textwidth]{27.png}

\subsection{Spurious Correlation}

Ratios induce spurious correlations. Solution: compositional data analysis.

x, y, and z are all statistically independent. 
The ratios induce high correlation between them.

Normalizing by a common number (such as in ratios) results in spurious correlation.

\subsection{Confounding Variables}
\begin{itemize}
    \item Beware of (latent) confounding variables
    \begin{itemize}
        \item Averaging over them can hide important trends
    \end{itemize}
\end{itemize}

\includegraphics[width=\textwidth]{28.png}

Political affiliation is a confounder. 
Belief in a scientific statement is conditioned on political affiliation.

\subsection{Correctly Normalizing}
\includegraphics[width=\textwidth]{29.png}

blue 
\begin{itemize}
    \item peaks between 0 and 1 and 23 and 25
    \item Not at certain temperatures with equivalent probability
    \item Looking at a joint measurement not a conditional measurement
\end{itemize}

red 
\begin{itemize}
    \item Normalized by the number of days in each temperature range
\end{itemize}

Fallacies
\begin{itemize}
    \item Don't look at a visualization before you make a hypothesis
\end{itemize}

\subsection{Myth of the Average}
\begin{itemize}
    \item High rate of plane crashes in the 40s
    \begin{itemize}
        \item Seats built for “average” pilot were non-adjustable!
    \end{itemize}
    \item Productivity of publications:
    \item We've averaged out the interesting behaviour of the data
\end{itemize}

\includegraphics[width=\textwidth]{30.png}

If plot early and later career productivity, a more
complex picture of productivity types emerges...

\includegraphics[width=\textwidth]{31.png}

x axis m1, y axis m2.
m1 is positive, m2 is negative.

We didn't see this many different behaviours because of averaging the data.

There are smaller subsets of trends that get shadowed by the majority trend

\subsection{Simpson's Paradox}
Possibly the most problematic issues in data analysis –
unobserved variables and/or sample bias can completely
change your interpretation and decision!

\includegraphics[width=\textwidth]{32.png}

How do resolve this?
– Which answer is correct depends on more knowledge

\end{document}