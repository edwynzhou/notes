\documentclass[11pt]{article}
\usepackage[utf8]{inputenc}	% Para caracteres en español
\usepackage{amsmath,amsthm,amsfonts,amssymb,amscd}
\usepackage{multirow,booktabs}
\usepackage[table]{xcolor}
\usepackage{fullpage}
\usepackage{lastpage}
\usepackage{enumitem}
\usepackage{fancyhdr}
\usepackage{mathrsfs}
\usepackage{wrapfig}
\usepackage{setspace}
\usepackage{calc}
\usepackage{multicol}
\usepackage{cancel}
\usepackage[retainorgcmds]{IEEEtrantools}
\usepackage[margin=3cm]{geometry}
\usepackage{amsmath}
\newlength{\tabcont}
\setlength{\parindent}{0.0in}
\setlength{\parskip}{0.05in}
\usepackage{empheq}
\usepackage{framed}
\usepackage[most]{tcolorbox}
\usepackage{xcolor}
\usepackage{graphicx}
\usepackage{listings}
% -- Basic formatting
\usepackage[utf8]{inputenc}
\usepackage[english]{babel}
\usepackage{times}
\setlength{\parindent}{8pt}
\usepackage{indentfirst}% -- Defining colors:
\usepackage[dvipsnames]{xcolor}
\definecolor{codegreen}{rgb}{0,0.6,0}
\definecolor{codegray}{rgb}{0.5,0.5,0.5}
\definecolor{codepurple}{rgb}{0.58,0,0.82}
\definecolor{backcolour}{rgb}{0.95,0.95,0.92}% Definig a custom style:
\lstdefinestyle{mystyle}{
    backgroundcolor=\color{backcolour},   
    commentstyle=\color{codepurple},
    keywordstyle=\color{NavyBlue},
    numberstyle=\tiny\color{codegray},
    stringstyle=\color{codepurple},
    basicstyle=\ttfamily\footnotesize\bfseries,
    breakatwhitespace=false,         
    breaklines=true,                 
    captionpos=t,                    
    keepspaces=true,                 
    numbers=left,                    
    numbersep=5pt,                  
    showspaces=false,                
    showstringspaces=false,
    showtabs=false,                  
    tabsize=2
}% -- Setting up the custom style:
\lstset{style=mystyle}
\lstset{
  style=mystyle,
  framexleftmargin=3.5mm,
  rulesepcolor=\color{black},
  linewidth=0.6\linewidth,
  xleftmargin=12pt,
  aboveskip=12pt,
  belowskip=12pt
}
\colorlet{shadecolor}{orange!15}
\parindent 0in
\parskip 12pt
\geometry{margin=1in, headsep=0.25in}
\theoremstyle{definition}
\newtheorem{defn}{Definition}
\newtheorem{reg}{Rule}
\newtheorem{exer}{Exercise}
\newtheorem{note}{Note}
\graphicspath{ {./images/} }
\begin{document}
\setcounter{section}{0}
\title{MIE223 Lecture Notes}

\thispagestyle{empty}

\begin{center}
{\LARGE \bf Data Science: Introduction}\\
{\large MIE223}\\
Winter 2025
\end{center}
\section{Introduction}
\subsection{Data Science}
Check out this screenshot:

\includegraphics[width=\textwidth]{1.png}
\begin{lstlisting}[language=Python]
    print("Hello World!")
\end{lstlisting}
\begin{note}
\textbf{Capital Letters refer to the accelerating reference frame \textit{S} while lowercase letters refer to the inertial reference frame S$_0$}
\end{note}
Picture a moving reference frame, \textit{S}, moving relative to S$_0$. Imagine in the the moving reference frame that a ball with mass, \textit{m} is being thrown. 
In order to consider the motion of the ball, the motion must be first considered in the inertial reference frame. 
\begin{equation}
F = m\ddot{r_0}
\end{equation}
Where r$_0$ is the ball's position relative to S$_0$. 

Now, by considering the motion of the ball in the accelerating frame, the ball position relative to \textit{S} is \textit{R}. (It's velocity is $\dot{R}$. 
Thus, relating \textit{R} to $r_0$, we have: 
\begin{equation}
\dot{r_0} = \dot{R} + V
\end{equation}
Newton's second law for the inertial reference frame by differentiate and multiplying by mass is:
\begin{equation}
F_{\text{inertial}} = -mA = -m\ddot{R}
\end{equation}
\newpage
\subsection{The Tides}
\begin{shaded}
\textbf{The Tidal Force} \newline
\begin{equation}
F_{tide} = -GM_mm(\frac{\hat{d}}{d^2}-\frac{\hat{d_0}}{d_0^2})
\end{equation}
Where:
\begin{equation*}
\begin{split}
G = \text{Gravitational Constant} \\
d = \text{Object's Position Relative to Moon} \\
d_0 = \text{Earth's Center Relative to the moon}\\
M_m = \text{Mass of the moon}
\end{split}
\end{equation*}
\end{shaded}
\end{document}