\documentclass[11pt]{article}
\usepackage[utf8]{inputenc}	% Para caracteres en español
\usepackage{amsmath,amsthm,amsfonts,amssymb,amscd}
\usepackage{multirow,booktabs}
\usepackage[table]{xcolor}
\usepackage{fullpage}
\usepackage{lastpage}
\usepackage{enumitem}
\usepackage{fancyhdr}
\usepackage{mathrsfs}
\usepackage{wrapfig}
\usepackage{setspace}
\usepackage{calc}
\usepackage{multicol}
\usepackage{cancel}
\usepackage[retainorgcmds]{IEEEtrantools}
\usepackage[margin=1cm]{geometry}
\usepackage{amsmath}
\newlength{\tabcont}
\setlength{\parindent}{0.0in}
\setlength{\parskip}{0.05in}
\usepackage{empheq}
\usepackage{framed}
\usepackage[most]{tcolorbox}
\usepackage{xcolor}
\usepackage{graphicx}
\usepackage{listings}
% -- Basic formatting
\usepackage[utf8]{inputenc}
\usepackage[english]{babel}
\usepackage{times}
\usepackage{caption}
\usepackage{subcaption}
\usepackage{placeins}
\setlength{\parindent}{0pt}
\usepackage{indentfirst}% -- Defining colors:
\usepackage[dvipsnames]{xcolor}
\definecolor{codegreen}{rgb}{0,0.6,0}
\definecolor{codegray}{rgb}{0.5,0.5,0.5}
\definecolor{codepurple}{rgb}{0.58,0,0.82}
\definecolor{backcolour}{rgb}{0.95,0.95,0.92}% Definig a custom style:
\lstdefinestyle{mystyle}{
    backgroundcolor=\color{backcolour},   
    commentstyle=\color{codepurple},
    keywordstyle=\color{NavyBlue},
    numberstyle=\tiny\color{codegray},
    stringstyle=\color{codepurple},
    basicstyle=\ttfamily\footnotesize\bfseries,
    breakatwhitespace=false,         
    breaklines=true,                 
    captionpos=t,                    
    keepspaces=true,                 
    numbers=left,                    
    numbersep=5pt,                  
    showspaces=false,                
    showstringspaces=false,
    showtabs=false,                  
    tabsize=2
}% -- Setting up the custom style:
\lstset{style=mystyle}
\lstset{
  style=mystyle,
  framexleftmargin=3.5mm,
  rulesepcolor=\color{black},
  linewidth=0.6\linewidth,
  xleftmargin=12pt,
  aboveskip=12pt,
  belowskip=12pt
}
\colorlet{shadecolor}{orange!15}
\parindent 0in
\parskip 1pt
\geometry{margin=1in, headsep=0.25in}
\theoremstyle{definition}
\newtheorem{defn}{Definition}
\newtheorem{reg}{Rule}
\newtheorem{exer}{Exercise}
\newtheorem{note}{Note}
\graphicspath{ {./images/} }
\linespread{0.75}
\begin{document}
\setcounter{section}{0}
\title{MIE223 Lecture Notes}

\thispagestyle{empty}

\begin{center}
{\LARGE \bf Sanner Said This Would be on the Midterm}\\
{\large MIE223}\\
Winter 2025
\end{center}

\section{Feature Analysis and Visualization}
\begin{verbatim}
  This will be on the midterm
  
  Univariate has discrete or numeric/continuous columns.
  
  Discrete: bar plot of frequency

  Numeric/Continuous: histogram, box plot, violin plot

  Discrete heatmap of MI, PMI can produce correlation between D and d

  Scatter plot, heatmap is good for C and C

  Box, violin, overlapping histogram plots for D and C

\end{verbatim}

\section{Basic NLP Text Processing}
\begin{itemize}
  \item Match dates from 2000 to 2025 (Reg-Ex)
  \begin{itemize}
    \item Only compound statements allowed
    \item Do not just list out all the options
    \item 20[0-1][0-9] | 202[0-5]
  \end{itemize}
\end{itemize}

\subsection{Exercise}
\begin{itemize}
  \item Write a regular expression to match dates
  \begin{itemize}
    \item November 9, 1989
    \item 17 December 1967
    \item 11-09-1989 (likely this form on midterm)
    \item 12/17/67
  \end{itemize}
  \item Write a regular expression to match time expressions
  \begin{itemize}
    \item Next Wednesday at noon
    \item Tomorrow morning
    \item Can't really as there's no general pattern
  \end{itemize}
\end{itemize}

\section{Advanced NLP Text Processing}
\begin{verbatim}
  no POS tagging tested on exams.
\end{verbatim}

\section{Time Series Data}

Sample midterm question: Would you use VADER to predict the stock market?

Answer: No, VADER is a sentiment analysis tool that is used to analyze text data. It is not used to predict the stock market.
Simple polarity sentiment is not complex enough to analyze the stock market

\end{document}